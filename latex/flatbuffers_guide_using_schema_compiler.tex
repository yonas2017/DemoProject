\chapter{Using the schema compiler}
\hypertarget{flatbuffers_guide_using_schema_compiler}{}\label{flatbuffers_guide_using_schema_compiler}\index{Using the schema compiler@{Using the schema compiler}}
Usage\+: \begin{DoxyVerb}flatc [ GENERATOR OPTIONS ] [ -o PATH ] [ -I PATH ] FILES...
      [ -- FILES...]
\end{DoxyVerb}
 The files are read and parsed in order, and can contain either schemas or data (see below). Data files are processed according to the definitions of the most recent schema specified.

{\ttfamily -\/-\/} indicates that the following files are binary files in Flat\+Buffer format conforming to the schema indicated before it.

Depending on the flags passed, additional files may be generated for each file processed\+:

For any schema input files, one or more generators can be specified\+:


\begin{DoxyItemize}
\item {\ttfamily -\/-\/cpp}, {\ttfamily -\/c} \+: Generate a C++ header for all definitions in this file (as {\ttfamily filename\+\_\+generated.\+h}).
\item {\ttfamily -\/-\/java}, {\ttfamily -\/j} \+: Generate Java code.
\item {\ttfamily -\/-\/kotlin} , {\ttfamily -\/-\/kotlin-\/kmp} \+: Generate Kotlin code.
\item {\ttfamily -\/-\/csharp}, {\ttfamily -\/n} \+: Generate C\# code.
\item {\ttfamily -\/-\/go}, {\ttfamily -\/g} \+: Generate Go code.
\item {\ttfamily -\/-\/python}, {\ttfamily -\/p} \+: Generate Python code.
\item {\ttfamily -\/-\/js}, {\ttfamily -\/s} \+: Generate Java\+Script code.
\item {\ttfamily -\/-\/ts}, {\ttfamily -\/T} \+: Generate Type\+Script code.
\item {\ttfamily -\/-\/php} \+: Generate PHP code.
\item {\ttfamily -\/-\/grpc} \+: Generate RPC stub code for GRPC.
\item {\ttfamily -\/-\/dart}, {\ttfamily -\/d} \+: Generate Dart code.
\item {\ttfamily -\/-\/lua}, {\ttfamily -\/l} \+: Generate Lua code.
\item {\ttfamily -\/-\/lobster} \+: Generate Lobster code.
\item {\ttfamily -\/-\/rust}, {\ttfamily -\/r} \+: Generate Rust code.
\item {\ttfamily -\/-\/swift} \+: Generate Swift code.
\item {\ttfamily -\/-\/nim} \+: Generate Nim code.
\end{DoxyItemize}

For any data input files\+:


\begin{DoxyItemize}
\item {\ttfamily -\/-\/binary}, {\ttfamily -\/b} \+: If data is contained in this file, generate a {\ttfamily filename.\+bin} containing the binary flatbuffer (or a different extension if one is specified in the schema).
\item {\ttfamily -\/-\/json}, {\ttfamily -\/t} \+: If data is contained in this file, generate a {\ttfamily filename.\+json} representing the data in the flatbuffer.
\item {\ttfamily -\/-\/jsonschema} \+: Generate Json schema
\end{DoxyItemize}

Additional options\+:


\begin{DoxyItemize}
\item {\ttfamily -\/o PATH} \+: Output all generated files to PATH (either absolute, or relative to the current directory). If omitted, PATH will be the current directory. PATH should end in your systems path separator, e.\+g. {\ttfamily /} or {\ttfamily \textbackslash{}}.
\item {\ttfamily -\/I PATH} \+: when encountering {\ttfamily include} statements, attempt to load the files from this path. Paths will be tried in the order given, and if all fail (or none are specified) it will try to load relative to the path of the schema file being parsed.
\item {\ttfamily -\/M} \+: Print make rules for generated files.
\item {\ttfamily -\/-\/strict-\/json} \+: Require \& generate strict JSON (field names are enclosed in quotes, no trailing commas in tables/vectors). By default, no quotes are required/generated, and trailing commas are allowed.
\item {\ttfamily -\/-\/allow-\/non-\/utf8} \+: Pass non-\/\+UTF-\/8 input through parser and emit nonstandard \textbackslash{}x escapes in JSON. (Default is to raise parse error on non-\/\+UTF-\/8 input.)
\item {\ttfamily -\/-\/natural-\/utf8} \+: Output strings with UTF-\/8 as human-\/readable strings. By default, UTF-\/8 characters are printed as \textbackslash{}u\+XXXX escapes."{}  -\/   $<$tt$>$-\/-\/defaults-\/json$<$/tt$>$ \+: Output fields whose value is equal to the default value     when writing JSON text.  -\/   $<$tt$>$-\/-\/no-\/prefix$<$/tt$>$ \+: Don\textquotesingle{}t prefix enum values in generated C++ by their enum     type.  -\/   $<$tt$>$-\/-\/scoped-\/enums$<$/tt$>$ \+: Use C++11 style scoped and strongly typed enums in     generated C++. This also implies $<$tt$>$-\/-\/no-\/prefix$<$/tt$>$.  -\/   $<$tt$>$-\/-\/no-\/emit-\/min-\/max-\/enum-\/values$<$/tt$>$ \+: Disable generation of MIN and MAX     enumerated values for scoped enums and prefixed enums.  -\/   $<$tt$>$-\/-\/gen-\/includes$<$/tt$>$ \+: (deprecated), this is the default behavior.                        If the original behavior is required (no include                        statements) use $<$tt$>$-\/-\/no-\/includes.$<$/tt$>$  -\/   $<$tt$>$-\/-\/no-\/includes$<$/tt$>$ \+: Don\textquotesingle{}t generate include statements for included schemas the     generated file depends on (\+C++ / Python).  -\/   $<$tt$>$-\/-\/gen-\/mutable$<$/tt$>$ \+: Generate additional non-\/const accessors for mutating     Flat\+Buffers in-\/place.  -\/   $<$tt$>$-\/-\/gen-\/onefile$<$/tt$>$ \+: Generate single output file for C\#, Go, and Python.  -\/   $<$tt$>$-\/-\/gen-\/name-\/strings$<$/tt$>$ \+: Generate type name functions for C++.  -\/   $<$tt$>$-\/-\/gen-\/object-\/api$<$/tt$>$ \+: Generate an additional object-\/based API. This API is     more convenient for object construction and mutation than the base API,     at the cost of efficiency (object allocation). Recommended only to be used     if other options are insufficient.  -\/   $<$tt$>$-\/-\/gen-\/compare$<$/tt$>$  \+:  Generate operator== for object-\/based API types.  -\/   $<$tt$>$-\/-\/gen-\/nullable$<$/tt$>$ \+: Add Clang \+\_\+\+Nullable for C++ pointer. or @\+Nullable for Java.  -\/   $<$tt$>$-\/-\/gen-\/generated$<$/tt$>$ \+: Add @\+Generated annotation for Java.  -\/   $<$tt$>$-\/-\/gen-\/jvmstatic$<$/tt$>$ \+: Add @\+Jvm\+Static annotation for Kotlin methods     in companion object for interop from Java to Kotlin.  -\/   $<$tt$>$-\/-\/gen-\/all$<$/tt$>$ \+: Generate not just code for the current schema files, but     for all files it includes as well. If the language uses a single file for     output (by default the case for C++ and JS), all code will end up in     this one file.  -\/   $<$tt$>$-\/-\/cpp-\/include$<$/tt$>$ \+: Adds an \#include in generated file  -\/   $<$tt$>$-\/-\/cpp-\/ptr-\/type T$<$/tt$>$ \+: Set object API pointer type (default std\+::unique\+\_\+ptr)  -\/   $<$tt$>$-\/-\/cpp-\/str-\/type T$<$/tt$>$ \+: Set object API string type (default std\+::string)     T\+::c\+\_\+str(), T\+::length() and T\+::empty() must be supported.     The custom type also needs to be constructible from std\+::string (see the     --cpp-\/str-\/flex-\/ctor option to change this behavior).  -\/   $<$tt$>$-\/-\/cpp-\/str-\/flex-\/ctor$<$/tt$>$ \+: Don\textquotesingle{}t construct custom string types by passing     std\+::string from Flatbuffers, but (char\texorpdfstring{$\ast$}{*} + length). This allows efficient     construction of custom string types, including zero-\/copy construction.  -\/   $<$tt$>$-\/-\/no-\/cpp-\/direct-\/copy$<$/tt$>$ \+: Don\textquotesingle{}t generate direct copy methods for C++     object-\/based API.  -\/   $<$tt$>$-\/-\/cpp-\/std CPP\+\_\+\+STD$<$/tt$>$ \+: Generate a C++ code using features of selected C++ standard.      Supported $<$tt$>$\+CPP\+\_\+\+STD$<$/tt$>$ values\+:     \texorpdfstring{$\ast$}{*} $<$tt$>$c++0x$<$/tt$>$ -\/ generate code compatible with old compilers (\+VS2010),     \texorpdfstring{$\ast$}{*} $<$tt$>$c++11$<$/tt$>$ -\/ use C++11 code generator (default),     \texorpdfstring{$\ast$}{*} $<$tt$>$c++17$<$/tt$>$ -\/ use C++17 features in generated code (experimental).  -\/   $<$tt$>$-\/-\/object-\/prefix$<$/tt$>$ \+: Customise class prefix for C++ object-\/based API.  -\/   $<$tt$>$-\/-\/object-\/suffix$<$/tt$>$ \+: Customise class suffix for C++ object-\/based API.  -\/   $<$tt$>$-\/-\/go-\/namespace$<$/tt$>$ \+: Generate the overrided namespace in Golang.  -\/   $<$tt$>$-\/-\/go-\/import$<$/tt$>$ \+: Generate the overrided import for flatbuffers in Golang.      (default is "{}github.\+com/google/flatbuffers/go"{}).  -\/   $<$tt$>$-\/-\/raw-\/binary$<$/tt$>$ \+: Allow binaries without a file\+\_\+indentifier to be read.     This may crash flatc given a mismatched schema.  -\/   $<$tt$>$-\/-\/size-\/prefixed$<$/tt$>$ \+: Input binaries are size prefixed buffers.  -\/   $<$tt$>$-\/-\/proto$<$/tt$>$\+: Expect input files to be .\+proto files (protocol buffers).     Output the corresponding .\+fbs file.     Currently supports\+: $<$tt$>$package$<$/tt$>$, $<$tt$>$message$<$/tt$>$, $<$tt$>$enum$<$/tt$>$, nested declarations,     $<$tt$>$import$<$/tt$>$ (use $<$tt$>$-\/\+I$<$/tt$>$ for paths), $<$tt$>$extend$<$/tt$>$, $<$tt$>$oneof$<$/tt$>$, $<$tt$>$group$<$/tt$>$.     Does not support, but will skip without error\+: $<$tt$>$option$<$/tt$>$, $<$tt$>$service$<$/tt$>$,     $<$tt$>$extensions$<$/tt$>$, and most everything else.  -\/   $<$tt$>$-\/-\/oneof-\/union$<$/tt$>$ \+: Translate .\+proto oneofs to flatbuffer unions.  -\/   $<$tt$>$-\/-\/grpc$<$/tt$>$ \+: Generate GRPC interfaces for the specified languages.  -\/   $<$tt$>$-\/-\/schema$<$/tt$>$\+: Serialize schemas instead of JSON (use with -\/b). This will     output a binary version of the specified schema that itself corresponds     to the reflection/reflection.\+fbs schema. Loading this binary file is the     basis for reflection functionality.  -\/   $<$tt$>$-\/-\/bfbs-\/comments$<$/tt$>$\+: Add doc comments to the binary schema files.  -\/   $<$tt$>$-\/-\/conform FILE$<$/tt$>$ \+: Specify a schema the following schemas should be     an evolution of. Gives errors if not. Useful to check if schema     modifications don\textquotesingle{}t break schema evolution rules.  -\/   $<$tt$>$-\/-\/conform-\/includes PATH$<$/tt$>$ \+: Include path for the schema given with     $<$tt$>$-\/-\/conform PATH$<$/tt$>$.  -\/   $<$tt$>$-\/-\/filename-\/suffix SUFFIX$<$/tt$>$ \+: The suffix appended to the generated     file names. Default is \textquotesingle{}\+\_\+generated\textquotesingle{}.  -\/   $<$tt$>$-\/-\/filename-\/ext EXTENSION$<$/tt$>$ \+: The extension appended to the generated     file names. Default is language-\/specific (e.\+g. "{}h"{} for C++). This     should not be used when multiple languages are specified.  -\/   $<$tt$>$-\/-\/include-\/prefix PATH$<$/tt$>$ \+: Prefix this path to any generated include     statements.  -\/   $<$tt$>$-\/-\/keep-\/prefix$<$/tt$>$ \+: Keep original prefix of schema include statement.  -\/   $<$tt$>$-\/-\/reflect-\/types$<$/tt$>$ \+: Add minimal type reflection to code generation.  -\/   $<$tt$>$-\/-\/reflect-\/names$<$/tt$>$ \+: Add minimal type/name reflection.  -\/   $<$tt$>$-\/-\/root-\/type T$<$/tt$>$ \+: Select or override the default root\+\_\+type.  -\/   $<$tt$>$-\/-\/require-\/explicit-\/ids$<$/tt$>$ \+: When parsing schemas, require explicit ids (id\+: x).  -\/   $<$tt$>$-\/-\/force-\/defaults$<$/tt$>$ \+: Emit default values in binary output from JSON.  -\/   $<$tt$>$-\/-\/force-\/empty$<$/tt$>$ \+: When serializing from object API representation, force      strings and vectors to empty rather than null.  -\/   $<$tt$>$-\/-\/force-\/empty-\/vectors$<$/tt$>$ \+: When serializing from object API representation, force      vectors to empty rather than null.  -\/   $<$tt$>$-\/-\/flexbuffers$<$/tt$>$ \+: Used with "{}binary"{} and "{}json"{} options, it generates      data using schema-\/less Flex\+Buffers.  -\/   $<$tt$>$-\/-\/no-\/warnings$<$/tt$>$ \+: Inhibit all warning messages.  -\/   $<$tt$>$-\/-\/cs-\/global-\/alias$<$/tt$>$ \+: Prepend $<$tt$>$global\+::$<$/tt$>$ to all user generated csharp classes and structs.  -\/   $<$tt$>$-\/-\/json-\/nested-\/bytes$<$/tt$>$ \+: Allow a nested\+\_\+flatbuffer field to be parsed as a     vector of bytes in JSON, which is unsafe unless checked by a verifier     afterwards.  -\/   $<$tt$>$-\/-\/python-\/no-\/type-\/prefix-\/suffix$<$/tt$>$ \+: Skip emission of Python functions that are prefixed     with typenames  -\/   $<$tt$>$-\/-\/python-\/typing$<$/tt$>$ \+: Generate Python type annotations  \+Additional g\+RPC options\+:  -\/   $<$tt$>$-\/-\/grpc-\/filename-\/suffix$<$/tt$>$\+: $<$tt$>$\mbox{[}\+C++\mbox{]}$<$/tt$>$ An optional suffix for the generated     files\textquotesingle{} names. For example, compiling g\+RPC for C++ with     $<$tt$>$-\/-\/grpc-\/filename-\/suffix=.\+fbs$<$/tt$>$ will generate $<$tt$>$\{name\}.\+fbs.\+h$<$/tt$>$ and     $<$tt$>$\{name\}.\+fbs.\+cc$<$/tt$>$ files.  -\/   $<$tt$>$-\/-\/grpc-\/additional-\/header$<$/tt$>$\+: $<$tt$>$\mbox{[}\+C++\mbox{]}$<$/tt$>$ Additional headers to include in the     generated files.  -\/   $<$tt$>$-\/-\/grpc-\/search-\/path$<$/tt$>$\+: $<$tt$>$\mbox{[}\+C++\mbox{]}$<$/tt$>$ An optional prefix for the g\+RPC runtime path.     For example, compiling g\+RPC for C++ with $<$tt$>$-\/-\/grpc-\/search-\/path=some/path$<$/tt$>$ will     generate the following includes\+:      $<$tt$>$cpp       \textbackslash{}\#include "{}some/path/grpcpp/impl/codegen/async\+\_\+stream.\+h"{}       \textbackslash{}\#include "{}some/path/grpcpp/impl/codegen/async\+\_\+unary\+\_\+call.\+h"{}       \textbackslash{}\#include "{}some/path/grpcpp/impl/codegen/method\+\_\+handler.\+h"{}       ... $<$/tt$>$  -\/   $<$tt$>$-\/-\/grpc-\/use-\/system-\/headers$<$/tt$>$\+: $<$tt$>$\mbox{[}\+C++\mbox{]}$<$/tt$>$ Whether to generate $<$tt$>$\textbackslash{}\#include \textbackslash{}$<$header\textbackslash{}$>$$<$/tt$>$     instead of $<$tt$>$\textbackslash{}\#include "{}header.\+h"{} for all headers when compiling g\+RPC for C++. For example, compiling g\+RPC for C++ with {\ttfamily -\/-\/grpc-\/use-\/system-\/headers} will generate the following includes\+:

{\ttfamily cpp \#include \texorpdfstring{$<$}{<}some/path/grpcpp/impl/codegen/async\+\_\+stream.\+h\texorpdfstring{$>$}{>} \#include \texorpdfstring{$<$}{<}some/path/grpcpp/impl/codegen/async\+\_\+unary\+\_\+call.\+h\texorpdfstring{$>$}{>} \#include \texorpdfstring{$<$}{<}some/path/grpcpp/impl/codegen/method\+\_\+handler.\+h\texorpdfstring{$>$}{>} ... }

NOTE\+: This option can be negated with {\ttfamily -\/-\/no-\/grpc-\/use-\/system-\/headers}.
\item {\ttfamily -\/-\/grpc-\/python-\/typed-\/handlers}\+: {\ttfamily \mbox{[}Python\mbox{]}} Whether to generate the typed handlers that use the generated Python classes instead of raw bytes for requests/responses.
\end{DoxyItemize}

NOTE\+: short-\/form options for generators are deprecated, use the long form whenever possible. 